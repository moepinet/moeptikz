%% documentation.tex
%% Copyright 2015 Stephan M. Guenther
%
% This work may be distributed and/or modified under the % conditions of the
% LaTeX Project Public License, either version 1.3 % of this license or (at
% your option) any later version. The latest version of this license is in
%   http://www.latex-project.org/lppl.txt
% and version 1.3 or later is part of all distributions of LaTeX version
% 2005/12/01 or later.
%
% This work has the LPPL maintenance status `maintained'.
%
% The current maintainer of this work is Stephan M. Guenther.
%
% This work consists of the files moeptikz.sty and documentation.tex.

\documentclass{scrartcl}

\usepackage[utf8]{inputenc}
\usepackage[T1]{fontenc}
\usepackage[english]{babel}
\usepackage{xspace}

\usepackage[shading=true]{moeptikz}
\usepackage{pgfmath}
\usepackage{siunitx}
\usepackage{calc}
\usepackage{inconsolata}

\usetikzlibrary{calc}
\usetikzlibrary{shapes}

\newcommand*{\TikZ}{Ti\textit{k}Z\xspace}
\newcommand*{\moeptikz}{mo\reflectbox{e}p\TikZ}
\newcommand*{\nodelabel}[1]{{\scriptsize\bfseries\ttfamily #1}}

\title{The \moeptikz{} package}
\subtitle{a library for typesetting computer networks}
\author{Stephan M. G\"unther}

\begin{document}

\maketitle

The \moeptikz{} package provides shapes for typesetting computer networks,
e.\,g.\@ shapes for hubs, switches, routers, and more.
It is in use since 2012 for the lecture materials of our basic course in
computer networks due to shortcomings of some commercial products and
(at that time) the lack of meaningful alternatives.
For the first official release many shapes were redesigned and a couple of new
features added, e.\,g.\@ complete anchor support and custom colors.
Currently the package provides the following shapes and commands:
\begin{itemize}\itemsep0pt
	\item \verb|hub|, \verb|switch|, \verb|router|, \verb|client|,
	\verb|server|, \verb|nuc|, \verb|messageclosed|, \verb|messageopen|
	\item \verb|\tikzextractx|, \verb|\tikzextracty|
\end{itemize}

\section{What it can do}
Did you ever try to build slides for a networking lecture from scratch?
If so, do you care about off-topic details such as using solely vector
graphics, perfect alignment of nodes, easy modification of hundreds of figures
when you decide to change fonts or colors, and doing all that without even
thinking about a pointing device?
Then you probably like what \moeptikz{} can do for you:

\begin{figure}[ht!]
\centering
\begin{tikzpicture}[x=1cm,y=1cm]
	\node[cloud,cloud puffs=13,minimum width=3cm,minimum height=1.5cm,
		draw=black,line width=1pt] (inet) at (4.25,0) {};
	\node at (inet) {\nodelabel{Internet}};
	\node[router] (r1) at (.5,0) {};
	\node[router,minimum size=5mm] (r2) at (inet.west) {};
	\node[switch] (s1) at (-2,0) {};
	\node[server,minimum size=12mm,label=below:\nodelabel{ swallow.moepi.net}]
		(server1) at (-2,-1.5) {};
	\node[server,minimum size=12mm,label=above:\nodelabel{dc.poorcompany.net}]
		(server2) at (-2,1.5) {};
	\node[server,minimum size=12mm,label=above:\nodelabel{8.8.8.8}]
		(server3) at (8,0) {};

	\foreach \x in {1,2,3} {
		\node[client,label=left:\nodelabel{PC\,\x}]
			(c\x) at ($(s1)+(-2.8,2.6-1.2*\x)$) {};
		\draw[black,line width=1pt] (c\x) -- (s1);
	}

	\draw[black,line width=1pt] (s1) -- (r1);
	\draw[black,line width=1pt] (s1) -- (server1);
	\draw[black,line width=1pt] (s1) -- (server2);
	\draw[black,line width=1pt] (inet) -- (server3);
	\draw[black,line width=1pt] (r1) -- (r2.west);

	\draw[dashed,black,line width=1pt] (r1.west) -- ++(-1,1)
		node[above right] {\nodelabel{eth0: 172.16.0.1/24}} -- ++(3,0);

	\draw[line width=1.5pt,-latex,black!40] (r1) to[bend right] 
		node[below,yshift=-2mm,messageclosed,minimum size=5mm] {} (server3);
\end{tikzpicture}
\caption{What \protect\moeptikz{} can (or cannot?) do for you}
\label{fig:moeptikz-intro}
\end{figure}


\section{What it cannot do}
You probably noticed the misalignment of \verb|PC2| (and its friend) in
Figure~\ref{fig:moeptikz-intro}.
Well, you might not have noticed it in the first place (I didn't either), but
now, as you know about that, it must drive you crazy.
The good news: it's your fault (in this particular case it is my fault, of
course).
\moeptikz cannot calculate coordinates for you.
If you are interested in writing \TikZ{} instead of pointing around you will
notice the difference:

\begin{figure}[ht!]
\centering
\begin{tikzpicture}[x=1cm,y=1cm]
	\node[cloud,cloud puffs=13,minimum width=3cm,minimum height=1.5cm,
		draw=black,line width=1pt] (inet) at (4.25,0) {};
	\node at (inet) {\nodelabel{Internet}};
	\node[router] (r1) at (.5,0) {};
	\node[router,minimum size=5mm] (r2) at (inet.west) {};
	\node[switch] (s1) at (-2,0) {};
	\node[server,minimum size=12mm,label=below:\nodelabel{ swallow.moepi.net}]
		(server1) at (-2,-1.5) {};
	\node[server,minimum size=12mm,label=above:\nodelabel{dc.poorcompany.net}]
		(server2) at (-2,1.5) {};
	\node[server,minimum size=12mm,label=above:\nodelabel{8.8.8.8}]
		(server3) at (8,0) {};

	\foreach \x in {1,2,3} {
		\node[client,label=left:\nodelabel{PC\,\x}]
			(c\x) at ($(s1)+(-2.8,2.6-1.3*\x)$) {};
		\draw[black,line width=1pt] (c\x) -- (s1);
	}

	\draw[black,line width=1pt] (s1) -- (r1);
	\draw[black,line width=1pt] (s1) -- (server1);
	\draw[black,line width=1pt] (s1) -- (server2);
	\draw[black,line width=1pt] (inet) -- (server3);
	\draw[black,line width=1pt] (r1) -- (r2.west);

	\draw[dashed,black,line width=1pt] (r1.west) -- ++(-1,1)
		node[above right] {\nodelabel{eth0: 172.16.0.1/24}} -- ++(3,0);

	\draw[line width=1.5pt,-latex,black!40] (r1) to[bend right] 
		node[below,yshift=-2mm,messageclosed,minimum size=5mm] {} (server3);
\end{tikzpicture}
\caption{What \protect\moeptikz{} can do for you if you can do your math}
\end{figure}


\section{Package options}

The only package option currently defined is \texttt{shading=<true|false>}.
If true (default), nodes are shaded to create the visual effect of a spotlight.
The shading has only a marginal impact on the resulting document size but may
be undesirable depending on the personal preference.


\section{The shapes}
\newcommand*{\presentnode}[2]{%
	\begin{tikzpicture}[x=1mm,y=1mm]
		\node[#1] (x) at (0,0) {};
		\foreach \x in {0,1,...,7} {
			\pgfmathparse{10*cos(\x/8*360)}
			\edef\xpos{\pgfmathresult}
			\pgfmathparse{10*sin(\x/8*360)}
			\edef\ypos{\pgfmathresult}
			\node[circle,fill=black!40,minimum width=1mm,inner sep=0pt]
				(p\x) at ($(x)+(\xpos,\ypos)$) {};
			\draw[line width=1pt,black!40] (x) -- (p\x);
		}
		\node[rectangle,draw=black!40,line width=.5pt,inner sep=0pt,minimum
			size=1cm] at (x) {};
		\node[#1,label=above:\color{red}$\times$] at (0,0) {};
		\node[#1,label=below:\color{red}$\times$] at (0,0) {};
		\node[#1,label=left:\color{red}$\times$]  at (0,0) {};
		\node[#1,label=right:\color{red}$\times$,font=\footnotesize\ttfamily]
			at (0,0) {\colorbox{white}{#2}};
	\end{tikzpicture}
}

Figure~\ref{fig:shapes} presents the different shapes as printed by
\verb|\node[<shape>] at ...| (the label font is typewriter here, but that's up
to you).
The default label shows the name of the corresponding shape.
The gray box around the shape is a unit square making the shapes bounding box
visible.
The for red crosses mark the label positions specified by a node option such as
\verb|label=<pos>:<labeltext>|.
The gray bullets demonstrate automatically calculated anchor positions.

If the node option \verb|minimum size| is omitted, nodes will be drawn in a
bounding box of $\SI{1}{\centi\meter}\,\times\,\SI{1}{\centi\meter}$.
Since not all shapes fill their bounding box, default label positions may vary
between nodes, e.\,g.\@ label positions of server and client shapes differ.
This may be undesirable proved to be better than having node labels too far
away from the visible shape.

\begin{figure}[ht!]
	\centering
	\presentnode{hub}{hub}
	\hskip5mm%
	\presentnode{switch}{switch}
	\hskip5mm%
	\presentnode{router}{router}
	\hskip5mm%
	\presentnode{client}{client}
	\vskip5mm
	\presentnode{server}{server}
	\hskip5mm%
	\presentnode{nuc}{nuc}
	\hskip5mm%
	\presentnode{messageclosed}{closedmessage}
	\hskip5mm%
	\presentnode{messageopen}{openmessage}
	\caption{Shapes currently defined}
	\label{fig:shapes}
\end{figure}


\subsection{Changing node colors}
There are very few occasions where you might want to highlight a specific node
by changing its color.
Or you are upset by the default mark color (black), i\,e.\@, the color of
arrows on top of some shapes.
You may change it at will:
all shapes support custom colors using the \verb|fill=<color>| option.
In addition to that, the message shapes also allow to change the line color
using the \verb|draw=<color>| option.

\vskip\baselineskip

\begin{minipage}{1.75cm}
\begin{tikzpicture}
	\node[router,fill=blue,draw=orange] at (0,0) {};
\end{tikzpicture}
\end{minipage}%
\begin{minipage}{\textwidth-1.75cm}%
\verb|\node[router,fill=blue,draw=orange] at (0,0) {};|
\end{minipage}

\vskip\baselineskip

\begin{minipage}{1.75cm}
\begin{tikzpicture}
	\node[messageclosed,fill=red!40,draw=red] at (0,0) {};
\end{tikzpicture}
\end{minipage}%
\begin{minipage}{\textwidth-1.75cm}%
\verb|\node[messageclosed,fill=red!40,draw=red] at (0,0) {};|
\end{minipage}

\vskip\baselineskip

\begin{minipage}{1.75cm}
\begin{tikzpicture}
	\node[messageopen,fill=orange!40,draw=orange] at (0,0) {};
\end{tikzpicture}
\end{minipage}%
\begin{minipage}{\textwidth-1.75cm}%
\verb|\node[messageopen,fill=orange!40,draw=orange] at (0,0) {};|
\end{minipage}

\vskip\baselineskip

The effect of the \verb|draw| option depends on the actual shape:
a \verb|router| will change the color of its marking (the arrows) while an
\verb|openmessage| or \verb|closedmessage| change their line color.
This behavior may seem inconsistent but is intended -- give it a try, or come
up with fix.



\subsection{Upcoming features}
There will be some additional shapes such as \verb|firewall| and
\verb|workstation| soon.
Feature requests are welcome.
And of course there will be many code updates, since the current pfg code is
all but nice is all but nice.

\section{The commands}
\moeptikz{} provides two commands to make your live with \tikz{} easier:
\begin{enumerate}\itemsep0pt
	\item \verb|\tikzextractx{(x,y)}{\<macroname>}|{}\\
		Extracts the \verb|x|-component of a coordinate and stores it
		\verb|\<macroname>| for later use.
	\item \verb|\tikzextracty{(x,y)}{\<macroname>}|{}
		Extracts the \verb|y|-component of a coordinate and stores it
		\verb|\<macroname>| for later use.
\end{enumerate}


\section{Known bugs and missing features}
\begin{itemize}\itemsep0pt
	\item \texttt{[bug]} Sometimes lines from a node to another node do not end
	at an automatically calculated anchor but at the node's center, causing the
	line to cross parts of a shape.
	Please report such cases (including your code) to \texttt{moepi@moepi.net}.
	\item \texttt{[missing]} \verb|openmessage| has a shiny sheet of shaded
	blue paper.
	At the moment there is no way to change its color (or turn off the shading)
	except for hacking my hacky pgf code.
\end{itemize}

\end{document}

